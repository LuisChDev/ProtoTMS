\documentclass[12pt]{extarticle}
\usepackage[utf8]{inputenc}
% \usepackage{cite}
\usepackage{biblatex}
\addbibresource{Biblio.bib}
\addbibresource{Pymes.bib}

\title{Evaluación de metaheurísticas e implementación de un producto de software
  para la solución de problemas de enrutamiento en las pequeñas y medianas
  empresas}
\author{Luis E. Chavarriaga Cifuentes}
\date{Agosto 2020}

\begin{document}

\maketitle

\section{Disciplina científica}
Ingeniería de Sistemas y Computación.

\section{Cobertura de la investigación}
El proyecto de investigación aquí propuesto se extiende a la evaluación de
metaheurísticas, técnicas modernas de optimización por computador para la
solución de varios tipos de problemas. Específicamente, se propone explorar su
utilidad en la solución de una familia de problemas conocida como Problema de
Enrutamiento de Vehículos (\textit{Vehicle Routing Problem}, o VRP), usado para
modelar problemas de transporte.

\begin{itemize}
\item El proyecto busca evaluar estrategias novedosas de optimización basadas en
  metaheurísticas, junto con una implementación tentativa. En ese sentido, no
  apunta a probar soluciones comerciales ya existentes para el VRP, como las
  presentes en WinQSB.
\item El proyecto se centrará en la evaluación de estas técnicas de optimización
  en la solución de la familia de problemas VRP. Dicho esto, no se prevee la
  realización de estudios sobre familias de problemas de optimización
  diferentes, como la de Job Stop Scheduling (JSS), que también tienen
  aplicaciones en el campo empresarial.
\item Si bien existen servicios de mapas alternativos a Google Maps (Apple Maps,
  Bing Maps y otros), el proyecto se enfocará en emplear las tecnologías de
  Google para hacer el procesamiento necesario de la información.
\end{itemize}

\section{Campo de la investigación}
Inteligencia Artificial.

\section{Antecedentes}
\subsection{Marco Teórico}
Supóngase que se tiene una empresa de domicilios con 5 repartidores. Cada uno
de ellos es capaz de llevar 5 pedidos. Se tienen unos 30 pedidos de diferentes
partes de la ciudad. ¿Qué pedidos deberán asignarse a qué repartidores, de modo
que el total de kilómetros recorridos por todos (o el tiempo total de recorrido)
sea el mínimo posible?

Este problema es conocido como Vehicle Routing Problem (VRP, por sus siglas en
inglés). Con el objeto de modelar condiciones más reales, se han ideado muchas
posibles variaciones del mismo, que tienen en cuenta el tráfico, la autonomía de
los repartidores (en caso, por ejemplo, de usar camiones para el reparto),
ventanas de tiempo, prioridad de los envíos, diferentes capacidades de carga, y
otras \cite{caceres-cruz_rich_2015}.

Matemáticamente, este problema implica encontrar una serie de caminos (tantos
como repartidores haya) que pasen por todos los destinos. Dado que el número de
posibles caminos entre ubicaciones crece factorialmente, nos encontramos en
frente de un problema de optimización factorial. Esta familia de problemas son
especialmente difíciles de resolver, debido al alto costo computacional que
implica calcular todas los recorridos posibles y su costo \cite{toth_paolo_vehicle_2002}.

Hoy día, existe una serie de técnicas computacionales, llamadas
\textit{metaheurísticas}, que, en lugar de calcular la solución exacta, buscan
hallar una aproximación lo suficientemente cercana a la real para ser útil.
Dichos métodos siguen reglas simples y generales, que no son específicas al
problema, sino que son inspiradas, usualmente, por procesos naturales
(son \textit{heurísticas}). Empleando metaheurísticas, es posible obtener
soluciones aproximadas al enrutamiento de envíos y transporte de carga y
personas en un tiempo razonable, haciéndolas viables para un negocio.

Para ello, es necesario un análisis preliminar del problema, de la naturaleza y
estructura de los datos y qué algoritmo metaheurístico funciona mejor en cada
uno de los casos. Una vez determinado, es posible elaborar un prototipo básico
para determinar las características que deberá tener el software para ser
utilizable en el mundo real.

\subsection{Estado del Arte}
Hay una extensa literatura acerca del VRP y sus aplicaciones en la industria
\cite{pillac_review_2013} \cite{gomez_metaheuristicas_2014}.
\cite{eksioglu_vehicle_2009} presenta una primera revisión de las publicaciones
con las palabras ``vehicle routing'' en el título, sujeto o abstract halladas en
las bases de datos digitales, realizadas entre 1959 y 2008. Asimismo, se propone
una primera clasificación de las variantes del mismo. Se observa incluso que la
frecuencia en que se publican artículos crece a un ritmo exponencial [pág.
1475]. Se halla que entre las fechas estipuladas se han publicado un total de
1021 artículos directamente aplicables al tema. \cite{braekers_vehicle_2016}
Realiza una revisión más reciente, de la literatura publicada entre 2009 y 2015,
empleando una taxonomía basada en \cite{eksioglu_vehicle_2009}, con ciertas
modificaciones. Se halla que, cumpliendo los criterios de selección
establecidos, unos 277 artículos han sido publicados entres las fechas. Ambas
revisiones del estado del arte hacen hincapié en la existencia de un cuerpo más
extenso aún de publicaciones centradas en problemas relacionados, como el de
empacado de vehículos, y su integración con el VRP. (Véanse
\cite{prodhon_survey_2014} \cite{pollaris_vehicle_2015} y
\cite{golden_inventory_2008} para revisiones de estos problemas relacionados, y
\cite{pillac_review_2013}, \cite{groer_consistent_2009} y
\cite{caceres-cruz_rich_2015} para revisiones de variantes específicas
publicadas recientemente.) 

Similarmente, debido a la proliferación de métodos de solución al problema, es
posible hallar una literatura dedicada a la comparación de dichos métodos.
\cite{breedam_comparing_2001} Compara la heurística de descenso, la búsqueda
tabú (BT) y el recocido simulado (RS) como metodologías para la solución del
VRP. El artículo concluye que si bien el descenso heurístico produce un
resultado más rápidamente, tanto el BT como el RS eventualmente logran un mejor
resultado, y que la diferencia entre los resultados de ambos rara vez supera el
4\%. Mientras tanto, escoger una buena heurística inicial influye
considerablemente en el tiempo que se tarda en llegar al resultado final. El
trabajo no incluye comparación a otros tipos de metaheurísticas.

\cite{asih_comparison_2017}, en un trabajo más reciente, compara un número más
extenso de metaheurísticas (\textit{Ant Colony Optimization} (ACO),
\textit{Particle Swarm Optimization} (PSO), Algoritmos genéticos (GA) y RS
(SA)), esta vez contra dos problemas reales. El primer caso compara el desempeño
de las metaheurísticas en generar las rutas para una empresa con 58 ubicaciones,
así como un escenario colaborativo en el que dos empresas comparten camiones y
puntos de reparto. los resultados muestran que el RS tiende a tener un buen
desempeño, sobretodo respecto al tiempo de computación, pero que esta ventaja
puede desvanecerse dado un número de locales más grande. Se hace notar que el
resultado del estudio tiene a diferir de los obtenidos en pruebas con datos
aleatorios: ``In hypothetical data, GA is outperform [\textit{sic}] compared to
other approaches, but in this empirical study, GA performs less than ACO and
even SA''. La sección de conclusiones nota que la evaluación de metaheurísticas
en estudios empíricos está aún abierta para exploración futura.

Finalmente, se han hecho también comparaciones sobre las metaheurísticas en
general. \cite{hussain_metaheuristic_2019} representa una investigación reciente
que observa, en particular, que el \textit{Particle Swarm Optimization} ha
recibido una amplia atención de la literatura en un amplio número de campos, y
que las metaheurísticas en su totalidad han recibido atención creciente como
método de solución a problemas muy diversos.

Existen hoy día soluciones de software para el problema, con aplicaciones
comerciales, como en \cite{erdogan_open_2017}, que aplican metaheurísticas al
problema. Sin embargo, no se observa un análisis detallado de la aplicabilidad
de las distintas metaheurísticas para resolver el problema.

Análisis de este tipo ya se han realizado, en torno a problemas de corte
comercial \cite{gavalas_survey_2014}. Se desea realizar un análisis similar
sobre los algoritmos ya implementados en varias herramientas, en particular, la
herramienta OR-Tools de Google \cite{noauthor_or-tools_nodate}, que permita
expresar la posibilidad de implementar una solución de este tipo de forma
accesible a pequeñas y medianas empresas en la región.

\section{Breve descripción del problema}

%%% esta sección pertenece al marco teórico (antecedentes)
%%% fin de la sección

La situación de las PYMEs en Colombia no es la mejor. En \cite[págs.
109-111]{r_situacion_2010}, se realiza un estudio acerca de la viabilidad del
sector en la economía colombiana, con resultados mixtos: de los 16 factores del
índice global de competitividad, Colombia había descendido de posición en 14 de
los mismos en el período de 2006 a 2008. Si bien las PYMEs constituían en 50\%
de las empresas exportadoras en el país (ya de por sí un número bajo,
considerando que constituyen el 96\% del total de empresas), sólo el 18,6\% de
las mismas logra vender con éxito en el exterior. \cite[pág.
110]{r_situacion_2010}, indica los doce elementos que más inciden en las
dificultades para las PYMEs en lograr mayor competitividad, de las cuales
resaltan los siguientes en el contexto de este documento: 
\begin{itemize}
\item Bajos niveles de innovación y absorción de tecnologías.
\item Deficiencias en la infraestructura de transporte y energía.
\item Rezago en penetración de tecnologías de información y conectividad.
\end{itemize}
La dificultad en la adquisición de nuevas tecnologías puede corregirse mediante
la puesta en disposición de productos funcionales a un precio asequible a las
PYMEs, que usualmente no pueden darse el lujo de contratar las soluciones
comerciales existentes, usualmente dirigidas a empresas más grandes. Las
deficiencias en infraestructura de transporte son difíciles de corregir, pero
apuntan a una necesidad considerable de optimizar esta parte del proceso allí
donde es posible. Por último, la introducción de un nuevo software de planeación
de transporte puede ayudar a estas empresas a formalizar los procesos por los
que colectan y procesan la información generada en sus actividades, con los
beneficios que esto implica.\cite[pág. 93]{premkumar_meta-analysis_2003}


En el período 2003-2006, se crearon 43716 nuevas empresas, pero se liquidaron
8593 empresas existentes, representando 3.3 billones de pesos en capital
acumulado, casi el 20\% respecto de las empresas formadas
\cite{espinosa_fracaso_2015}\cite[págs. 14-21]{bogota_dinamica_2006}. De acuerdo
a \cite[págs. 13-14]{vega_pymes_2011}\cite[págs. 23-24]{bogota_dinamica_2006}, tres de las principales
causas de cierre de las entidades son:
\begin{itemize}
\item Dificultades de las empresas para adaptarse a los cambios tecnológicos.
\item Baja utilización en la capacidad instalada o un alto índice de
  inventarios.
\item Reducción del capital y aumento significativo de los pasivos.
\end{itemize}
El primer punto apunta a la dificultad que pueden tener muchas empresas en
adoptar nuevas tecnologías, aspecto crítico para mantenerse competitivas, y que
puede ser causado por los altos precios propios de soluciones comerciales. Esto
toca en el tercer punto, ya que la introducción de pasivos a las finanzas de una
PYME le puede resultar fatal. Estos costos adicionales tienen al menos parte de
su origen en los costos de transporte de mercancía: En un estudio realizado por
\cite[pág. 6]{gaytan_logistica_2017}, las entidades comerciales encuestadas
daban una cifra de alrededor del 20\% en gastos de logística, que se contrastan
con el 8\% observado en países como Estados Unidos. Por otra parte, se ha
observado que la aplicación de métodos de optimización permite obtener ahorros
de entre 5 y 20\% en los costos totales de transporte
\cite{toth_paolo_vehicle_2002}.


\section{Justificación}
La economía colombiana, al igual que en el resto del mundo, ha visto dos
importantes transformaciones: El crecimiento del sector secundario, encabezado
por la industria textil y pesada \cite{naranjo_industria_nodate}, y más tarde,
el del sector terciario, con servicios como la banca y el comercio exterior
tomando cada vez más preponderancia \cite{urrutia_perspectivas_1990} . En el
contexto actual, la expansión vertiginosa de este último ha hecho que extertos
en el tema consideren una división entre los aspectos más tradicionales del
mismo, como las ventas y el transporte, y los servicios basados en la
información, el conocimiento y la tecnología, muchas veces llamado sector
cuaternario y cuya existencia queda respaldada por el impulso a la ``economía
naranja'' que se le ha dado a nivel nacional
\cite{noauthor_econominaranja_nodate}. Dicha economía de la información permite
un incremento en la productividad de todos los anteriores sectores, al optimizar
procesos, reducir costos, e incrementar la calidad de los resultados.

Un servicio transversal a los sectores primario, secundario y terciario es el
del transporte, crítico para el movimiento de materias primas, productos
manufacturados y personas, y el funcionamiento correcto de estas industrias
\cite{r_situacion_2010}. Históricamente, la esquematización del transporte y
la optimización de rutas ha sido un proceso de alta complejidad, del cual
depende la rentabilidad de muchos negocios en Colombia. Si bien existen
herramientas con este propósito, el costo de las mismas es prohibitivo para
muchas pequeñas y medianas empresas (PYMEs) en el país.

En este sentido, el desarrollo reciente de técnicas de optimización basadas en
metaheurísticas ofrece una oportunidad para economizar este proceso y ofrecerle
a este segmento de la economía los medios para aumentar sus márgenes y mejorar
sus proyecciones financieras \cite{gomez_metaheuristicas_2014}. Asimismo, la
creación local de un producto de software tiene consecuencias positivas para la
expansión de la economía naranja y la consolidación de la industria del software
en Colombia y sus servicios.

Este proyecto propone la aplicación de metaheurísticas en la construcción de una
aplicación de software para la optimización de recorridos de transporte.

\section{Objetivos}

\subsection{Objetivo general}
Implementar y evaluar un producto de software que permita, a través de
metaheurísticas, optimizar rutas de transporte para PYMEs.

\subsection{Objetivos específicos}
\begin{itemize}
\item Modelar de forma realista las características de los problemas de
  enrutamiento de las empresas para su representación en forma de parámetros
  ajustables en los algoritmos metaheurísticos, de manera que su desempeño
  se vea reflejado en situaciones reales.
\item Caracterizar el desempeño de distintas metodologías basadas en
  metaheurísticas para la optimización de procesos, en particular, la solución
  del llamado Problema del enrutamiento de vehículos (\textit{Vehicle Routing
  Problem}, o VRP) y sus variaciones.
\item Diseñar e implementar un prototipo de producto de software con la
  capacidad de generar soluciones de enrutamiento para pequeñas y medianas
  empresas.
\item Evaluar y validar el prototipo resultante y sus especificaciones.
\end{itemize}

\printbibliography[title={Bibliografía}]

\end{document}

% Local Variables:
% ispell-local-dictionary: "castellano"
% End:
